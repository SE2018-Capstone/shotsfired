\documentclass[11pt, oneside]{article}   	% use "amsart" instead of "article" for AMSLaTeX format
\usepackage{geometry}                		% See geometry.pdf to learn the layout options. There are lots.
\geometry{letterpaper}                   		% ... or a4paper or a5paper or ... 
%\geometry{landscape}                		% Activate for for rotated page geometry
%\usepackage[parfill]{parskip}    		% Activate to begin paragraphs with an empty line rather than an indent
\usepackage{graphicx}				% Use pdf, png, jpg, or eps§ with pdflatex; use eps in DVI mode
								% TeX will automatically convert eps --> pdf in pdflatex		
\usepackage{amssymb}

\title{SE 464 Project Proposal -- Shots Fired}
\author{Sam Maier (scmaier), Addison Keizer (wcakeize), \\Dhruv Lal (d2lal), Shiranka Miskin (samiskin)}
%\date{}							% Activate to display a given date or no date
\let\endtitlepage\relax
\begin{document}
\maketitle
\section{The Project}
%\subsection{}
\hspace*{5mm} Our project is a 2-4 player 2D multiplayer shooter game. The multiplayer will be online. To build this application we will use Typescript, a modified Javascript. Node.js will be our backend. \\

This project is interesting for both technical and market availability reasons. There is an opening in the market for a well made, free and fun web application that has easy inter-friend multiplayer capability through the internet. There are a ton of games that are very popular due to their multiplayer component.   \\

Technology wise this project is even more interesting. The software must be constructed in such a way that the multiplayer appears seamless to the users, which is a significant challenge when done from the beginning and very interesting. Furthermore this gives us the opportunity to learn a new language in Typescript. This project will also have interesting software architecture because there will need to be a well structured set of software components to represent game objects and the different players and online multiplayer tasks. The selection of Node.js as backend means that it will be easier to duplicate game logic between clients and server and we will be able to reuse code. \\

This project makes sense in the chosen environment for a variety of reasons. Our goal for this is to make a fluid and responsive game in on the web to give the users the easiest possible conditions to join into a game with their friends. No install - no setup time. Mobile or desktop both have large mental barriers to entry and our aim is to tear those down. Not to mention that it is completely environment agnostic Mac, PC, Linux - it works for all. In terms of the software directly, this project will give us a chance to learn new technologies all the while working on a challenging problem, the implementation of multiplayer.

\clearpage

\section{Functional Requirements}
\begin{enumerate}
\item Is 2-4 player, each player plays on their own computer. 
\begin{enumerate}
\item Player will be able to control their avatar using keyboard and/or mouse.
      They would press W to move forward, A and D to move left and right
      respectively, and S to move downward.  They would aim their weapon using the mouse.
\end{enumerate}
\item Capability for user to join a matched game
\item Capability for user to play with known people in an isolated environment
\item Is a 2D arena style game
\begin{enumerate}
\item 2-4 human players have their own character, and must navigate a small 2Dmap
\item Numerical based health.
\item Each player has an equal finite amount of health
\item Each player has a weapon with infinite ammo
\end{enumerate}
\item Can handle multiple simultaneous games
\end{enumerate}

\section{User Scenarios}


\textbf{First Scenario}\\\\

Consider a user that is with 2 other friends and they are just trying to pass the time or have some fun. They each have their own laptops. Shots Fired is a 2-4 player game and is a game that everyone can play together. This satisfies the need of getting everyone involved in the same activity. To create a game with friends, one person would create a game and choose the option to play with friends. In this case, they will receive a code or url that their friends can input and use to join the same group. This way, friends can play together and not have to worry about having other people in their game. \\\\

Each player will control a character on a small 2D map.  Their view would show
a top-down view of the map, with their character in the center. Using the keys
WASD, they would move their character around, and they will use the mouse
to aim and fire their weapon. If hit, they lose health. When health is 0 they
lose. A user would be able to complete a game in under 5 minutes so it is very
quick and easy to play anytime. Due to the fact they are playing with friends,
there is a bit of a competition between them, however, they are mainly playing
for fun to pass the time. 

\clearpage

\noindent\textbf{Second Scenario}\\\\

Consider a user that is alone. They load the game. They do not have any
friends, but they still want a way to relax and play a simple game with others.
This game will support this use case. The user would open the game, then select
the "Quick play" option.  The system would then place the user along with other
users who selected "Quick play" in an automatically created instance.  Once the
instance has 4 players or there are more than one player and a time limit is
reached, the game will begin.  The game itself will be the same as what is
described in the first user scenario.  This scenario also allows for a user of
a more competitive nature since they are not playing with friends and can try
their hardest to do their best.




\section{Non Functional Properties}
\begin{enumerate}
\item
Performance:  Output is rendered without noticeable lag between players if players are on
computers with a reasonable internet connection. \\\\
It is important to have no noticeable lag since that will affect the user
experience. If players experience lag, they will be less likely to play the
game again since lag harms the gameplay. We want users to continue playing
the game in the future. No lag will result in a much better playing
experience in game.
\item
Usability: Very easy to join a game. Low barrier to entry.\\\\
Having a low barrier to entry is important so that anyone can join the game. By
using a code, a user can easily play a match with friends. By joining a
match without a code, a user can easily play with other random people so
both game modes allow for easy entry.
\item
Simplicity:  Simple controls and objective, easy to pick up and play\\\\
Part of the low barrier to entry is how easy it is to learn how to
play the game and be able to have fun with it.  The rules of the game
should be simple and intuitive enough to immediately grasp what
one should do.
\item
Compatibility: Compatible with Chrome 52+.\\\\
The compatibility with Chrome 52+ is important so that we have access to any
new features that come out in browser. Chrome 52 is one of the most recent
versions of Chrome that was released a few months ago so it should be
commonly used right now. In addition, Chrome will update on its own which
means that majority of users will be able to play the game and make use of
new Chrome features if available.

\end{enumerate}

\clearpage

\includegraphics[scale=.085]{mockup.jpg}\\
A mockup of the in-game screen with the blocks acting as obstacles, circles as players, and the objects they're holding as guns.


\end{document}  
