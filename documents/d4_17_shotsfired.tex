\documentclass[12pt]{report}
\usepackage{geometry}                		% See geometry.pdf to learn the layout options. There are lots.
\geometry{letterpaper}                   		% ... or a4paper or a5paper or ... 
\usepackage{hyperref}
\usepackage{graphicx}				% Use pdf, png, jpg, or eps§ with pdflatex; use eps in DVI mode
\usepackage{fullpage}
\setlength{\parskip}{1em}
\author{Shiranka Miskin, Sam Maier, Dhruv Lal, Addison Keizer}
\title{Shots Fired - Architecture and Design}
\begin{document}
\maketitle

\tableofcontents
\chapter{Architecture}
\section{Requirements}
\subsection{Functional Requirements}
\subsection{Non-functional Attributes}
\section{Challenges}

The main challenge that is shared amongst many games is that the complexity
involved is much higher than most other types of applications that may only need
to display information and handle simple interactivity.  Games such as ours must
be able to handle multiple entities acting simultaneously, performing a variety
of different tasks such as movement, performing actions like firing weapons, and
having other entities be able to affect them such as being damaged.  Inputs
through potentially multiple devices must be constantly polled and applied, and
the game must be continually drawn and updated.

This complexity is increased by a potentially large factor when online
multiplayer is added. Now a single game must be synchronized between machines
despite the significant delay between updates due to having to transmit data
over an internet connection.  Significant effort was put into attempting
different ways of dealing with this latency.

Another challenge that applies to games more than it does to other applications
is performance.  The entire game must be able to process the inputs, update the
game, and display its results at a sufficient framerate to appear smooth to
users, the standard target framerate being 60 frames per second.

\section{Final Architecture}
\subsection{Client/Server}
There are multiple ways of solving the issue of synchronizing a game between
multiple computers (which will be referred to as "clients") over a network.  The
differences lie in what type of data is being transmitted from each client, and
which machine has "authority" about what the true state of the game is.

In Shots Fired, the server runs an entire copy of the game, which allows it to
act as the "source of truth" to keep all clients in sync.  The only information
that clients send to the server are their inputs, and then the server sends a
master copy of the game state to all the clients.  This allows for two very
important benefits, but has serious implications on the code.  To understand the
benefits, consider a scenario where the server did not have to run the game
logic, and it was only ran on the clients:  
\begin{enumerate}
    \item There is now the question of what the "true" state of the game is.
        Due to latency being in the hundreds of miliseconds, there is no
        guarantee that each client will process the game in exactly the same
        way.  Small differences in each client's state can result in a
        "butterfly effect", where each successive change simply increases just
        how desynchronized the clients are.  A way to get around this while
        still avoiding having to run the game on the server is to have one
        computer act as the host, and their state is taken to be the true state
        of the game.  The issue that arises here is the concept of "host
        advantage", where the player acting as a host now has 0 latency, while
        all other players have hundreds of miliseconds of latency.  
    \item Since the clients are given a measure of authority on the game state,
        it now becomes much easier to hack our game.  In order to run on
        browsers, the game is written in JavaScript, which, due to it being a
        scripted language and the debuggers that come built in to browsers, is
        easier for a user to make changes to than with other environments.  A
        player could theoretically simply send events to say "eliminate all
        other players" and they would win.  This is impossible in our
        architecture, as the only valid messages the server can process are
        snapshots of user input, such as which buttons they are pressing.
\end{enumerate}

\subsection{Node.js and Typescript}
One of the most significant implications of having to run the game logic on both
the client and the server is that in order for the code to be shared, it must be
able to be run on both platforms.  Since the game runs on the browser it must
be ran on JavaScript, or a language which can compile to JavaScript.  The forces
our choice of server technology to be Node.js, which allows servers to be
developed using JavaScript.

Significant effort was made to attempt a concurrent architecture on the server,
however after numerous road blocks due to the details of Node, it was deemed
to be infeasible.  JavaScript as a language is designed to be single threaded,
and Node.js simply does not accomodate for communication and sharing between
processes.

Out of the many language options which compile to JavaScript, TypeScript was
chosen in order to gain the many benefits of static type checking.  Being able
to assert the types of objects was even more significant due to the use of the
blackboard pattern.

\subsection{Blackboard}
In order to keep all clients in sync, the server must be able to send
sufficient information on the game state over a websocket connection to the
clients.  This data must therefore be serializable, and must be sent over at a fast
enough rate, ideally 60 ticks per second just like the client, as to not allow
discrepancies between clients to build up.  To accomodate this, Shots Fired
implements a Blackboard style, where the entirety of the game state is stored in
a single JavaScript object.  The server is then able to easily forward updates
to the clients by simply sending this JavaScript object over, no potentially
costly serialization or deserialization is necessary.

There were other benefits of this architectural decision to store the game state in a
single JavaScript object.  In order to have the client replace its state with
the same state as the server, all that is necessary is to replace the value of
the state variable in the client with the data received from the server.  Having
it be a plain JavaScript object is also a great aid when it comes to debugging,
as a simple \texttt{console.log(game)} call will output the entire object in an
interactive and easily viewable format in the browser's debug menu.  It becomes
incredibly simple to view the entire state of the app at some point in time and
be able to figure out the cause of the issue at hand.

\subsection{Pipe and Filter}
Shots Fired involves multiple entities which need to undergo a various set of
changes every frame of the game.  To process these in a structured manner, a
pipe and filter approach was implemented.  These updates to a game were kept
in strictly separate stages in order to reduce coupling between these
functionalities.  The original game object would be passed through these stages
to determine the state of the game for the next frame.

\subsection{Events}
In a game such as this, each entity may have various ways of affecting other
entities, such as a bullet damaging a character or two character colliding.  In
order to handle this very large number of possible interactions, an event-based
approach was selected.  Game entities are not permitted to affect change in any
other entities other than through events.  Once all individual updates are
applied to every entity, the resulting events are applied afterwards in their
own stage.  Entity interaction becomes strictly decoupled from any individual
interactions such as simple movement physics.

\chapter{Design}

\end{document}
