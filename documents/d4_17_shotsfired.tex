\documentclass[12pt]{report}
\usepackage{geometry}                		% See geometry.pdf to learn the layout options. There are lots.
\geometry{letterpaper}                   		% ... or a4paper or a5paper or ... 
\usepackage{graphicx}				% Use pdf, png, jpg, or eps§ with pdflatex; use eps in DVI mode
\usepackage{fullpage}
\setlength{\parskip}{1em}
\author{Shiranka Miskin, Sam Maier, Dhruv Lal, Addison Keizer}
\title{Shots Fired - Architecture and Design}
\begin{document}
\maketitle

\tableofcontents
\chapter{Architecture}
\section{Requirements}
\subsection{Functional Requirements}
\subsection{Non-functional Attributes}
\section{Challenges}

The main challenge that is shared amongst many games is that the complexity
involved is much higher than most other types of applications that may only need
to display information and handle simple interactivity.  Games such as ours must
be able to handle multiple entities acting simultaneously, performing a variety
of different tasks such as movement, performing actions like firing weapons, and
having other entities be able to affect them such as being damaged.  Inputs
through potentially multiple devices must be constantly polled and applied, and
the game must be continually drawn and updated.

This complexity is increased by a potentially large factor when online
multiplayer is added. Now a single game must be synchronized between machines
despite the significant delay between updates due to having to transmit data
over an internet connection.  Significant effort was put into attempting
different ways of dealing with this latency.

Another challenge that applies to games more than it does to other applications
is performance.  The entire game must be able to process the inputs, update the
game, and display its results at a sufficient framerate to appear smooth to
users, the standard target framerate being 60 frames per second.

\section{Final Architecture}
\subsection{Client/Server}
There are multiple ways of solving the issue of synchronizing a game between
multiple computers (which will be referred to as "clients") over a network.  The
differences lie in what type of data is being transmitted from each client, and
which machine has "authority" about what the true state of the game is.  One
implementation would be to have each client run the game logic and send
events of their actions to a server, which would then forward those actions to
the other clients.  The issue with this type of implementation is that there is
a question of what the "true" game state actually is, since there are simply
multiple copies of it on multiple machines.  

\subsection{Node.js, Typescript, and Phaser}
\subsection{Blackboard}
\subsection{Pipe and Filter}
\subsection{Events}

\chapter{Design}

\end{document}
